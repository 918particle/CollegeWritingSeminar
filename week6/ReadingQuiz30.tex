\documentclass{article}
\usepackage{graphicx}
\usepackage[margin=1.5cm]{geometry}
\usepackage{amsmath}
\usepackage{hyperref}
\hypersetup{
    colorlinks=true,
    linkcolor=blue,
    filecolor=blue,      
    urlcolor=blue,
    pdftitle={Overleaf Example},
    pdfpagemode=FullScreen,
    }

\begin{document}

\title{Week 13 Writing Activity: Citations in Essays and Articles}
\author{Prof. Jordan C. Hanson (INTD100)}

\maketitle

\section{The Scientific Attitude and the Social Sciences}

As a summary reflection on the social sciences and \textit{the scientific attitude}, let us watch the TED talk by famous non-fiction author Malcom Gladwell entited ``Choice, happiness and spaghetti sauce'' \cite{HH}.  Answer the following questions below after we finish the video.

\begin{enumerate}
\item Consider the following list of typical flaws associated with social science studies, and recall how the company that manufactured the tomato sauce \textit{Prego} was thinking about their product.  Next to each item, write an example of a potential error \textit{Prego} was making before they revised their product.
\begin{itemize}
\item \textit{Too much theory}:
\item \textit{Lack of experimental data}:
\item \textit{Fuzzy concepts}:
\item \textit{Ideological infection}:
\item \textit{Cherry picking}:
\item \textit{Lack of data sharing}:
\item \textit{Lack of replication}:
\item \textit{Lack of data sharing}:
\item \textit{Questionable causation}:
\end{itemize}
\item Note that Dr. Howard Moskowitz is responsible for the idea of \textit{market segments}.  You may know this term as a \textit{market demographic}, or a group of customers known to a company for their specific set of preferences.  Describe how Howard Moskowitz constructed his study of spaghetti sauces.  (a) How did he vary the recipe? (b) How was the data collected?  (c) In which way was the data analyzed: (1) Looking for the optimal sauce recipe that satisfied the highest number of market segments, or (2) allowing the data to guide the researcher to market segments?  \\ \vspace{2cm}
\item Redesign the Diet Pepsi study discussed early in the talk to account for the following traits: (a) sweetness (b) fizziness, and (c) tartness (think: amount of citrus). \\ \vspace{2cm}
\end{enumerate}

\begin{thebibliography}{10}
\bibitem{HH} Gladwell, Malcom. ``Choice, happiness and spaghetti sauce,'' \emph{TED Talk}, Jamuary 2007. \url{https://youtu.be/iIiAAhUeR6Y}.
\end{thebibliography}

\end{document}
