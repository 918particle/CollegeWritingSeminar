\documentclass{article}
\usepackage{graphicx}
\usepackage[margin=1.5cm]{geometry}
\usepackage{amsmath}
\usepackage{url}

\begin{document}

\title{Midterm 1: College Writing Seminar}
\author{Prof. Jordan C. Hanson}

\maketitle

\textbf{Assigned:} October 28th, 2022 at 11:00 am.  \textbf{Due:} October 31st, 2022 at 9:00 pm. Submit all answers in one PDF document, and submit this PDF on Moodle under the midterm 1 submission link.

\section{Unit 1: Concise Writing 1}

\begin{enumerate}
\item \textit{Using the delete button.} For the sentences below, re-write them more concisely. \\ \\
\textbf{Create an edited version in your document.}
\begin{enumerate}
\item Knowing the orbits of the stars around the center of the galaxy, scientists use the orbits to calculate the mass of the object at the center of the galaxy.  The object has the mass that is so large the mass has to be of a black hole.
\item Epidemiologists use a parameter called the reproduction parameter, $R0$, which is the number of new infections resulting from one new infected person.
\item According to the Newton's Laws of motion, things that have different masses and different shapes would still acceleratae downward at the same rate when dropped.
\end{enumerate}
\item \textit{Creating an outline.} Create an outline of the following set of ideas, such that it describes how to determine optimal tomato growing conditions.  Use the outline to write a well-organized paragraph describing the experiment. Submit \textit{both} the paragraph and the outline.  \\ \\
\textbf{Create a paragraph in your document.} 
\begin{itemize}
\item Ten tomato seedlings are obtained
\item A patch in the garden is reserved with space for all ten
\item A photo-sensor can be used to determine the light level at each spot in the patch
\item Each tomato plant is given a different amount of water per day
\item This whole process is done during the summer when the amount of sunshine is maximized
\end{itemize}
\end{enumerate}

\section{Unit 2: Concise Writing 2}

\begin{enumerate}
\item \textit{Hierarchy of detail and outlines}.  Choose from any of the 4 topics from slide 4 of the Week 2 Lecture Notes.  Select 3-4 sources online and use them to create an outline with the appropriate hierachy of details covering the subject.  Submit the outline and a 200 word summary of the subject, written concisely and without ambiguous words or phrasing. Properly cite your sources. \\ \\
\textbf{Add the work to your document.}
\end{enumerate}

\section{Unit 3: Technical Description 1}

\begin{enumerate}
\item \textit{Removing ambiguous words.}  In the following sentences, remove or replace ambiguous words. \\ \\
\textbf{Write the new sentences in your own document.} 
\begin{itemize}
\item When born, the baby was fairly heavy and really long.
\item The baby grew really fast, by the time she was 1 year old, she was a lot longer.
\item Radio transmission took a long while between the Earth and the Moon.
\item A hiker walked the full 60 km trail in 4 days, making her average speed moderate.
\end{itemize}
\item \textit{Spatial and temporal detail, perspective.}  Recall the exercise we performed in class, in which we wrote our favorite recipe.  In this exercise, explain to the reader from where you are gathering the ingredients, \textit{and} the recipe.  Thus, the result should be a tract of writing that would enable someone to prepare the dish using your kitchen and pantry. Notice how this requires you to pay attention to both time and space. \\ \\
\textbf{Write a paragraph in your own document.} 
\end{enumerate}

\section{Unit 4: Technical Description 2}

\begin{enumerate}
\item \textit{Convert to passive voice.} \\ \\
\textbf{Re-write the paragraph in your own document.} \\ \\
I measured the acceleration due to Earth's gravity, $g$, with a pendulum.  First, I measured the length of my pendulum to be 20 cm.  Second, I hung my pendulum straight down and displaced the bob 5 cm to my right.  I released the pendulum and recorded the number of times it returned to the same position as it swung back and forth for one minute.  I calculated that it returned to its original position every 0.90 seconds.  I inserted my results into the formula predicted by Newton's Laws.  The result for $g$ was 9.81 m/s$^2$.
\item \textit{Rearrange the sentences to have the proper hierarchy of detail.} \\ \\
\textbf{Re-write a paragraph in your own document.} \\ \\
The trials were conducted in a room with no air conditioning, and therefore no air flow.  The average horizontal distance bacteria travel after a person sneezes was measured.  First, a sample of 20 infected people was gathered.  The category of dishes with the largest colonies were the ones corresponding to 8.0 meters.  Third, once each subject felt the urge to sneeze, the subject was required to aim the sneeze down the line without covering their mouth.  The height of each subject was required to be within 6 inches of 5 feet 6 inches tall.  Second, petri dishes were arranged in 0.5 meter intervals out to 10.0 meters on the floor in front of the subject.  Fourth, bacterial colonies were allowed to grow in the dishes for one week under ideal conditions.  These results inform the epidemiology of spreading bacteria. The results show that when a person sneezes, it is possible to spread infection to someone who happens to be 8.0 meters away.
\item \textit{Rearrange the sentences to have the proper hierarchy of detail, and convert to passive voice.  Remove ambiguous words, and make the writing more concise.} \\ \\
\textbf{Re-write a paragraph in your own document.} \\ \\
Using a diagram of the forces, we show that the tangent of the angle is the friction coefficient.  I measured the coefficient of friction to be 0.095.  The tangent of the angle is measured many times.  The average friction coefficient is 0.095.  We placed an eraser on a meter stick.  We increased the angle between the meter stick and the table.  We measured the angle with a protractor.  We increased the angle until the eraser slides off the meter stick.  Using a diagram of the forces, we show that the tangent of the angle is the friction coefficient.   The standard deviation of the coefficient was 0.05.  A future idea for an experiment is to change the temperature of the eraser and determine if the friction coefficient depends on temperature.
\end{enumerate}

\end{document}