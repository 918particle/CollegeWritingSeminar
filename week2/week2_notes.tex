\documentclass{beamer}
\usetheme{metropolis}
\usepackage{graphicx}
\usepackage{amsmath}
\usepackage{tcolorbox}
\title{College Writing Seminar (INTD100): Week 2 Notes}
\author{Jordan Hanson}
\institute{Whittier College Department of Physics and Astronomy}

\begin{document}
\maketitle

\section{Summary}

\begin{frame}{Summary}
\textbf{Week 2}: \textit{Concise writing II:} In Week 2, we will focus on reading a piece of science writing, and creating your \textit{own} writing that tailors the story to a particular audience.
\begin{itemize}
\item Exercises: work in teams to produce a piece of science writing intended for a broad audience that weaves together information from a variety of sources: (a) a TED talk (b) a scientific journal article (c) and resources like Wikipedia and Google Scholar
\item Homework: writing a post designed for social media about a piece of science that grabs the attention of a wide audience, and attempts to convince that audience that the science is interesting
\item Exploration topic: Black hole observations
\end{itemize}
\end{frame}

\section{Group Project}

\begin{frame}{Group Project: Collaborative Science Writing}
\textbf{\alert{Instructions}}: Systematically random groups (next slide)
\begin{enumerate}
\item Choose one of the following four topics on the following slides
\item Choose a \textit{corresponding author} who will create a Google Document and share it with the others
\item Select a number of sources pertaining to the topic
\begin{itemize}
\item TED talks
\item Scientific journals from \url{arXiv.org} and \url{scholar.google.com}
\item Sources located on Wikipedia
\end{itemize}
\item Create a map or outline with Coggle.it or other tool that summarizes recent discoveries
\item Write a 1 page, single-spaced, 12 point font summary ($\approx 800-1000$ words)'
\item \alert{Bonus points: including a separate bibliography, correctly formatted}
\end{enumerate}
\end{frame}

\begin{frame}{Group Project: Collaborative Science Writing}
\small
\begin{columns}[T]
\begin{column}{0.5\textwidth}
\begin{enumerate}
\item Grace	Cooper ... Group A
\item Zack	Duhala ... Group B
\item Juan	Estrada ... Group C
\item Jusraunaq	Farmahan ... Group D
\item Mateo	Gomez ... Group A
\item Elise	Hansen ... Group B
\item Wyatt	Killien ... Group C
\item Kyle	Miller ... Group D
\end{enumerate}
\end{column}
\begin{column}{0.5\textwidth}
\begin{enumerate}
\item Eliot	Moser ... Group A
\item Rudy	Reyes ... Group B
\item Nick	Reynolds ... Group C
\item Paulina Valdez ... Group D
\item Andrea Wainwright ... Group A
\item Natasha Waldorf ... Group B
\item Emma Walston ... Group C
\end{enumerate}
\end{column}
\end{columns}
\end{frame}

\begin{frame}{Group Project: Collaborative Science Writing}
\small
\underline{Science Topics}
\begin{enumerate}
\item Event Horizon Telescope and the First Picture of a Black Hole
\begin{itemize}
\item What is long-baseline interferometry?
\item What are the properties of the black hole observed?
\end{itemize}
\item LIGO, Virgo and the First Neutron Star - Black Hole Merger
\begin{itemize}
\item What is a neutron star?
\item What is a black hole neutron stare merger?
\end{itemize}
\item IceCube Neutrino Observatory
\begin{itemize}
\item What is a neutrino?
\item What is IceCube Neutrino Observatory and where is it located?
\item What major discoveries have they made so far?
\end{itemize}
\item Anything related to COVID-19 and the pandemic
\begin{itemize}
\item How is the rate of spread quantified?
\item How fatal is the virus, and how does this vary for people?
\end{itemize}
\end{enumerate}
\end{frame}

\begin{frame}{Group Project: Collaborative Science Writing}
\small
\underline{Source classes}
\begin{enumerate}
\item TED talks: surprisingly useful at the start: \url{https://www.ted.com/talks/katie_bouman_how_to_take_a_picture_of_a_black_hole?utm_campaign=tedspread&utm_medium=referral&utm_source=tedcomshare}
\item Wikipedia has more sources: Event Horizon Telescope \url{https://en.wikipedia.org/wiki/Event_Horizon_Telescope}, references section leads you to \url{arXiv.org} with the exact set of references to journals
\item \url{arXiv.org} and \url{scholar.google.com}, as previously discussed, leads to journals
\item NewScientist, Space.com, Scientific American, etc.
\end{enumerate}
\end{frame}

\begin{frame}{Group Project: Collaborative Science Writing}
\begin{figure}
\includegraphics[width=0.65\textwidth]{figures/blackhole.pdf}
\caption{\label{fig:black} Black hole observations and Event Horizon Telescope.}
\end{figure}
\end{frame}

\begin{frame}{Group Project: Collaborative Science Writing}
\underline{Sources to Outline: Which details to cut?}
\begin{enumerate}
\item Kill your darlings
\item Hierarchy of detail: which \textit{level} of detail to cut?
\item Using analogies to replace finest details, equations
\item Cite or quote experts where appropriate
\end{enumerate}
\end{frame}

\begin{frame}{Group Project: Collaborative Science Writing}
\begin{figure}
\includegraphics[width=0.65\textwidth]{figures/blackhole2.pdf}
\caption{\label{fig:black2} Black hole observations and Event Horizon Telescope, take 2.}
\end{figure}
\end{frame}

\section{Conclusion}

\begin{frame}{Summary}
\textbf{Week 2}: \textit{Concise writing II:} In Week 2, we will focus on reading a piece of science writing, and creating your \textit{own} writing that tailors the story to a particular audience.
\begin{itemize}
\item Exercises: work in teams to produce a piece of science writing intended for a broad audience that weaves together information from a variety of sources: (a) a TED talk (b) a scientific journal article (c) and resources like Wikipedia and Google Scholar
\item Homework: writing a post designed for social media about a piece of science that grabs the attention of a wide audience, and attempts to convince that audience that the science is interesting
\item Exploration topic: Black hole observations
\end{itemize}
\end{frame}

\bibliographystyle{plain}
\section{Bibliography}

\bibliography{bibfile}

\end{document}
