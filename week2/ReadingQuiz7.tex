\documentclass{article}
\usepackage{graphicx}
\usepackage[margin=1.5cm]{geometry}
\usepackage{amsmath}
\usepackage{hyperref}

\begin{document}

\title{Week 4 Writing Activity: Hierarchy of Detail and the Abstract}
\author{Prof. Jordan C. Hanson (INTD100)}

\maketitle

\section{Concise Writing I: Hierarchy of Detail Exercise, 2}

One application within technical writing that requires a structured hierarchy of detail is an \textit{abstract.}  An abstract is a single paragraph that summarizes an entire work as concisely as possible.  A good abstract (a) convinces the reader the work is interesting, (b) convinces the reader the work makes sense, and (c) contains the key ideas or findings of the work.  Consider the map in Fig. \ref{fig:1} of a work describing the solution to ``childbed feaver'' discovered by Dr. Semmelweiz in 1840s Vienna.  Write an abstract that summarizes the ideas in Fig. \ref{fig:1} as quickly as possible. \\ \vspace{6cm}

\begin{figure}[hb]
\centering
\includegraphics[width=0.9\textwidth]{figures/semmelweiz.pdf}
\caption{\label{fig:1} A map of the ideas surrounding the scientific discovery of the cause of maternal mortality in the 19th Century.}
\end{figure}

\end{document}
