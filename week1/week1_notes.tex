\documentclass{beamer}
\usetheme{metropolis}
\usepackage{graphicx}
\usepackage{amsmath}
\usepackage{tcolorbox}
\title{College Writing Seminar (INTD100): Week 1 Notes}
\author{Jordan Hanson}
\institute{Whittier College Department of Physics and Astronomy}

\begin{document}
\maketitle

\section{Summary}

\begin{frame}{Summary}
\begin{enumerate}
\item \textbf{Week 1}: \textit{Concise writing I:} In Week 1, we will focus on creating concise writing that eliminates extraneous words and sentences from your writing.
\begin{itemize}
\item Exercises: distilling complex scientific articles into shorter tracts of writing.
\item Exercises: reading popular scientific journal articles and discussing them in small groups
\item Exercises: Practice using analogies in communicating difficult or abstract scientific thoughts
\item Homeworks: Practice quotation and paraphrasing of experts in writing
\item Homeworks: practice descriptive detail by providing the reader with the correct details such that they understand something complex
\item Exploration topic: gravitational waves
\end{itemize}
\end{enumerate}
\end{frame}

\section{Introductions, Technology Check}

\begin{frame}{Introduction}
\begin{enumerate}
\item Professor Jordan Hanson (Physics and Astronomy)
\item Syllabus, journal
\item \alert{\textbf{Coggle.it}} accounts (mind-mapping and writing organization for concise writing)
\item Breakout rooms test
\item Simple exercise with Google Docs/Screenshare
\item \textbf{Summer reading assignment I}
\begin{itemize}
\item Discuss the essay by James Baldwin in breakout rooms
\item Discuss answers to exercises
\item Sort out logistical issues
\end{itemize}
\item Homework: read article regarding discovery of gravitational waves by the LIGO/Virgo collaborations
\item \alert{Theme for Module 1:} \textit{Keep it simple.}
\end{enumerate}
\end{frame}

\section{Summer Reading Assignment}

\begin{frame}{Summer Reading Assignment - Screen Share, Breakout Rooms}
\small
\begin{enumerate}
\item Create a Google Document in your shared Whittier College Google Drive. All first-year students should be
given a Google Drive shared space by Whittier College.
\item Create a enumerated list (1., 2., 3. etc.). Each item is to a separate mini-assignment, and each could take
between a few minutes and an hour.
\item Item 1: This essay is in a narrative style, and is broken into several parts. Summarize part I in one paragraph, using 120 words or fewer. Note: The purpose of this exercise is to complete the summary in no more than a
certain number of words. What key moments stand out in your mind? What is the central realization of the
author at the end of the section?
\item Item 2: Repeat exercise 1, but instead use only twenty words. What happens in part I of the essay?
\end{enumerate}
\end{frame}

\begin{frame}{Summer Reading Assignment - Screen Share, Breakout Rooms}
\small
\begin{enumerate}
\item Item 3: Consider part 1.2I. Construct a tract of writing that a) defines and explains the poison metaphor the
author describes at the funeral of his father, b) identifies the author’s “cure” or cures for the poison, and c)
provides several pieces of supporting evidence for the identification of the author’s cure.
\item Item 4: Consider the final part of the essay, when the author describes the fight in the Hotel Braddock. Write
a tract of between 200-400 words on the author’s treatment of evidence and facts. What does the author have
to say about the importance of facts about the fight to the people in that neighborbood? Notice the author’s
writing takes on the tone of a reporter regarding the ensuing riot. What facts stand out regarding the outcome?
\end{enumerate}
\end{frame}

\section{Homework Article 1}

\begin{frame}{LIGO Discovery of Gravitational Waves}
\textit{Homework article 1.0:} \\
\url{https://www.ligo.caltech.edu/news/ligo20160211} \\
\begin{itemize}
\item Please begin reading this article this week
\item We will use it for in-class exercises regarding concise writing and article-mapping
\end{itemize}
\end{frame}

\section{Concise Writing 1.1: Use the eraser}

\begin{frame}{Concise Writing, 1.1}
How do we create concise writing?  Why is it important to be concise when writing for a scientific audience?
\begin{enumerate}
\item How...\textit{Be alone.}
\item How...\textit{Create an outline, mind-map or other plan}
\item How...\textit{\textbf{Use the eraser.}}  An eraser is a tool.  So is the delete button.
\item Why...Concise writing means fewer opportunities to be misunderstood
\item Why...Concise writing is easier to read
\item Why...Concise writing communicates abstract ideas into concrete form
\end{enumerate}
\end{frame}

\begin{frame}{Concise Writing, 1.1}
\textbf{Google Docs, Week 1}. \textit{Distill the following paragraph into a more conscise one.} \\ \vspace{0.25cm}
\textit{In order to measure the coefficient of friction of the rubber eraser, we placed the eraser on top of the plastic cover of the textbook and began to tilt it.  We continued to record the angle as we tilted the textbook.  Eventually, the eraser began to slide down the textbook.  We recorded that angle as well, and we calculated the tangent of that angle.  This was our first result for the coefficient of friction.  Next, we repeated the experiment many times and calculated an average coefficient of friction from the many trials.}
\alert{Pre-recorded content:} editing this down.
\end{frame}

\begin{frame}{Concise Writing, 1.1}
\textbf{Google Docs, Week 1}. \textit{Distill the following paragraph into a more conscise one.} \\ \vspace{0.25cm}
\textit{We set out to measure the specific heat of water by passing a wire through a bucket of water.  The water was 1000 milliliters and the water had a current of 2 amps.  The Joule heating equation tells us how many Joules of heat are produced in the wire which is radiated into the water.  We recorded the temperature of the water over time and recorded it next to the time in seconds.  We calcuated according to the Joule heating equation how many Joules per second was put into the water by the wire.  By comparing the rise in temperature of the water and the current we can calculate the conversion coefficient, which is the specific heat of water.}
\alert{Pre-recorded content:} editing this down.
\end{frame}

\section{Concise Writing 1.2: Outlines and Mind-Maps}

\begin{frame}{Concise Writing 1.2}
\begin{figure}
\centering
\includegraphics[width=0.7\textwidth]{figures/MindMap1.png}
\caption{\label{fig:mm1} A typical outline or mind-map for a scientific article for a wider audience.  For example, a summary of a field of research via Scientific American.}
\end{figure}
\end{frame}

\begin{frame}{Concise Writing 1.2}
\begin{figure}
\centering
\includegraphics[width=0.75\textwidth]{figures/MindMap2.png}
\caption{\label{fig:mm2} A typical outline or mind-map for a college lab report with an abstract.}
\end{figure}
\end{frame}

\begin{frame}{Concise Writing 1.2}
\begin{figure}
\centering
\includegraphics[width=0.75\textwidth]{figures/MindMap3.png}
\caption{\label{fig:mm3} What I often find in first-year essays in my physics courses.}
\end{figure}
\end{frame}

\begin{frame}{Concise Writing 1.2}
\textbf{Coggle.it, Week 1}.
\begin{enumerate}
\item Write a $\approx 200$ word summary of what you read in the homework article.
\item Using Coggle.it, or the tool of your choice, create a mind-map or outline of the homework article regarding gravitational waves.
\item Think about your \textit{nodes} and \textit{connections}.  Is there a way to simplify the outline?
\item Write a $\approx 200$ word summary of what you read in the homework article, based on your outline or mind map.
\item Compare the two summaries.
\end{enumerate}
\end{frame}

\begin{frame}{Concise Writing 1.2}
\small
Be alone. \\ \vspace{0.5cm}
\alert{\textbf{The essay on Moodle} Solitude and Leadership} by William Deresiewicz.
\end{frame}

\begin{frame}{Concise Writing 1.2}
\begin{figure}
\includegraphics[width=0.7\textwidth]{figures/Solitude_and_Leadership.pdf}
\caption{\label{fig:leadership}  A map of the essay \textit{Solitude and Leadership.}}
\end{figure}
\end{frame}

\begin{frame}{Concise Writing 1.2}
\begin{enumerate}
\item What is the central theme of the West Point article about leadership and solitude?
\item What does it mean to work alone in the context of leadership?  Leaders, by definition, are around other people.
\item Reflect on your writing \textit{process.}  If you cannot identify what process or processes you undertake to complete your writing, that's normal.  Write down a list of steps in your journal that constitute your writing process.
\end{enumerate}
\end{frame}

\begin{frame}{Concise Writing 1.2}
\textbf{An example process: my own for longer reports.}
\begin{enumerate}
\item Make an outline, with enumeration and bullet points.
\item Walk away and think about something else
\item Re-do the outline, and ensure it has concrete goals and sub-goals.
\item \textit{Identify any important graphics or tables, and work on those first.}
\item Begin writing:
\begin{itemize}
\item Write the introduction first
\item Write the next section next, while cutting down the introduction
\item Write the third section next, while cutting down the second section...
\item Re-examine the whole structure periodically.
\end{itemize}
\end{enumerate}
\end{frame}

\section{Concise Writing 1.3}

\begin{frame}{Concise Writing 1.3}
\begin{enumerate}
\item Exercises: Practice using analogies in communicating difficult or abstract scientific thoughts
\item Exercises: Practice quotation and paraphrasing of experts in writing
\item Exercises: hierarchy of detail
\end{enumerate}
\end{frame}

\begin{frame}{Concise Writing 1.3: Analogies}
\begin{figure}
\includegraphics[width=9cm]{figures/waves.jpg}
\caption{\label{fig:waves} A graphic of spacetime in 2D when two black holes orbit.}
\end{figure}
\end{frame}

\begin{frame}{Concise Writing 1.3: Analogies}
\begin{enumerate}
\item Exercise 1: Write three to four sentences describing the gravity waves that flow outward from the black hole pair, then \textit{edit it down.}
\item Exercise 2: Write three to four sentences describing the gravity waves that flow outward from the black hole pair, \textbf{using an analogy or metaphor}, then \textit{edit it down.} Potential analogies:
\begin{itemize}
\item Water waves due to a pebble dropped into a still pool
\item Marble spiraling down a funnel
\item One of your own choosing...
\end{itemize}
\end{enumerate}
\end{frame}

\begin{frame}{Concise Writing 1.3: Citations}
\alert{\textbf{Citations}:} Citing experts and technical references must be kept to a consistent style.  Below are some examples of citations in scientific journals. \\ \vspace{0.25cm}
``Recently, the LIGO and Virgo collaborations published demonstrated the existence of black holes $\approx 100$ times the mass of the sun \cite{10.1103/PhysRevLett.125.101102}.'' \\ \vspace{0.25cm}
``The authors of \cite{10.1103/PhysRevLett.118.221101} concluded that black holes larger than just a few times the mass of the sun do exist.'' \\ \vspace{0.25cm}
(More rare) ``Figure 7 of \cite{10.1109/iwem.2014.6963645} displays a mm-wave 16-element phased-array antenna system...''
Notice the following:
\begin{itemize}
\item Structure and placement
\item When is a citation required
\end{itemize}
\end{frame}

\begin{frame}{Concise Writing 1.3: Citations}
\begin{enumerate}
\item Exercise A: Go to \url{arXiv.org} and browse for a scientific paper that you think looks interesting.  You can find mostly physics, math, computer science and engineering papers, but also things like quantitative biology.  Once you find one, search for it on Google Scholar.
\item Exercise B: Once you find the webpage for your paper, determine the following: 
\begin{itemize}
\item The journal title, volume, and number (or year)
\item The lead authors or collaboration
\item The article title
\end{itemize}
Determine whether it is a \textit{conference proceeding} or a \textit{peer-reviewed article.}
\end{enumerate}
\end{frame}

\begin{frame}{Concise Writing 1.3: Hierarchy of Detail}
\alert{\textbf{Hierarchy of details}}: the organization of writing can make it more concise with \textit{better clarity.}
\begin{enumerate}
\item Good concise writing contains \textit{details}, but not \textit{too many} details.
\item The same is true for presentations.
\item Details form a hierarchy, just like the outline or map of a piece of writing.
\end{enumerate}
\end{frame}

\begin{frame}{Concise Writing 1.3: Hierarchy of Detail}
\begin{figure}
\includegraphics[width=6cm]{figures/PicketG.pdf}
\caption{\label{fig:picket} The classic setup for measurement og $g$, the acceleration due to gravity.}
\end{figure}
\end{frame}

\begin{frame}{Concise Writing 1.3: Hierarchy of Detail}
\textbf{Acceleration due to gravity}, well-constructed with appropriate detail. \\ \vspace{0.2cm}
The acceleration due to gravity was measured with the following technique.  A plastic slat with alternating black and transparent sections was dropped in front of an optical sensor.  The black and transparent sections were the same length.  When the optical sensor was blocked by the black sections, the time was recorded.  The difference in subsequent times decreases as the plastic slat accelerates.  The acceleration due to gravity was deduced from the increasingly short time-differences and the length of the black sections.  The acceleration was $10.0 \pm 0.4$ m/s$^2$.
\end{frame}

\begin{frame}{Concise Writing 1.3: Hierarchy of Detail}
\textbf{Acceleration due to gravity}, hierarchy of detail out of balance. \\ \vspace{0.2cm}
The acceleration due to gravity was measured, and the result was $10.0 \pm 0.4$ m/s$^2$.  A plastic slat with alternating black and transparent sections was dropped in front of an optical sensor.  The acceleration due to gravity was deduced from the increasingly short time-differences and the length of the black sections.  When the optical sensor was blocked by the black sections, the time was recorded.   The black and transparent sections were the same length.  The difference in subsequent times decreases as the plastic slat accelerates\footnote{Note: these are the \textit{exact same sentences} as in the prior paragraph.  Is the experiment as easily understood as before?}.
\end{frame}

\begin{frame}{Concise Writing 1.3: Hierarchy of Detail}
\begin{figure}
\includegraphics[width=8cm]{figures/soccer.jpeg}
\caption{\label{fig:soccer} \textit{Projectile motion} refers to the arch or quadratic curve objects make when moving through the Earth's gravity with some initial velocity.}
\end{figure}
\end{frame}

\begin{frame}{Concise Writing 1.3: Hierarchy of Detail}
\textbf{Projectile motion}, well-constructed with appropriate detail. \\ \vspace{0.2cm}
According to Newton's Laws, the distance an object should travel horizontally is the initial velocity squared, divided by the acceleration due to gravity, if the object is launched initially at a 45 degree angle with respect to horizontal.  A slingshot was tested on a pebble, and pulled back to a consistent length before releasing the pebble.  With an optical sensor, the velocity of the pebble was measured to be 10 meters per second.  When launched at a 45 degree angle, the pebble should travel 10.2 meters according to the prediction.  When launched, the pebble traveled 9.1 meters, for a fractional error of 10 percent.
\end{frame}

\begin{frame}{Concise Writing 1.3: Hierarchy of Detail}
\textbf{Projectile motion}, hierarchy of detail out of balance. \\ \vspace{0.2cm}
A slingshot was tested on a pebble, and pulled back to a consistent length before releasing the pebble.  When launched at a 45 degree angle, the pebble should travel 10.2 meters according to the prediction.  According to Newton's Laws, the distance an object should travel horizontally is the initial velocity squared, divided by the acceleration due to gravity, if the object is launched initially at a 45 degree angle with respect to horizontal.  With an optical sensor, the velocity of the pebble was measured to be 10 meters per second.  When launched, the pebble traveled 9.1 meters, for a fractional error of 10 percent.
\end{frame}

\begin{frame}{Concise Writing 1.3: Hierarchy of Detail}
Several things to notice:
\begin{itemize}
\item The details flow from \textit{general} to \textit{specific}.  When possible, measurments or variables are quoted later.  The setup or general idea occurs in the first sentence.
\item If a key variable is referenced, it is defined \textit{before} used.
\item Explaination of detail is still concise, so the rules regarding concise writing still apply.
\end{itemize}
\end{frame}

\begin{frame}{Concise Writing 1.3: Hierarchy of Detail}
\begin{figure}
\includegraphics[width=6cm]{figures/spring.jpeg}
\caption{\label{fig:spring} The force a spring exerts is proportional to the distance from its equilibrium point.  Create a paragraph describing how you would measure the force exerted by the spring, with the details in the correct order.}
\end{figure}
\end{frame}

\begin{frame}{Concise Writing 1.3: Hierarchy of Detail}
\alert{\textbf{Equations.}}  Equations can be particularly troubling when attempting to keep details in the proper hierarchy.  Consider the following equation describing projectile motion:
\begin{equation}
y(t) = -\frac{1}{2}gt^2 + v_{i,y} t + y_i
\end{equation}
Used in a piece of scientific writing, the equation would be introduced by defining each variable in words, and then presented in mathematical form.
\end{frame}

\begin{frame}{Concise Writing 1.3: Hierarchy of Detail}
\alert{\textbf{Good example}}: \\
Let $y(t)$ represent the height of an object above a given surface.  Let $g$ represent the acceleration due to gravity.  Let $t$ be the time, assuming that the beginning of the motion corresponds to $t = 0$.  The height of the object at $t = 0$ is $y_i$, and $v_{i,y}$ is the vertical velocity at $t = 0$.  According to Newton's Laws, $y(t)$ is
\begin{equation}
y(t) = -\frac{1}{2}gt^2 + v_{i,y} t + y_i
\end{equation}
\end{frame}

\begin{frame}{Concise Writing 1.3: Hierarchy of Detail}
\alert{\textbf{Average example}}: \\
Let $y(t)$ represent the height of an object above a given surface.  According to Newton's Laws, $y(t)$ is
\begin{equation}
y(t) = -\frac{1}{2}gt^2 + v_{i,y} t + y_i
\end{equation}
The variable $g$ represents the acceleration due to gravity.  The variable $t$ is the time.  It is assumed that the beginning of the motion corresponds to $t = 0$.  The height of the object at $t = 0$ is $y_i$, and $v_{i,y}$ is the vertical velocity at $t = 0$. \\ \vspace{0.5cm}
\textit{The reader is left wondering what all these symbols mean.}  They know what you want to say, but they can't see the purpose of all the letters in the formula unless you prepare them to understand.
\end{frame}

\begin{frame}{Concise Writing 1.3: Hierarchy of Detail}
\alert{\textbf{Bad example}}: \\
According to Newton's Laws, $y(t)$ for projectile motion is
\begin{equation}
y(t) = -\frac{1}{2}gt^2 + v_{i,y} t + y_i
\end{equation} \\ \vspace{0.5cm}
\textit{Technically, this is true.  However, no detail is provided, and so the reader either accepts it or moves on, not necessarily knowing what you mean.} \\ \vspace{0.5cm}
\textbf{It is challenging to avoid this in a longer work.  It's super easy to just stick in the formula you need and move forward.}
\end{frame}

\begin{frame}{Concise Writing 1.3: Hierarchy of Detail}
\begin{figure}
\includegraphics[width=3.5cm]{figures/yoda1.jpg}
\includegraphics[width=3.5cm]{figures/yoda2.jpg} \\
\includegraphics[width=3.5cm]{figures/yoda3.jpg}
\includegraphics[width=3.5cm]{figures/yoda4.jpg}
\caption{\label{fig:yodas} This is what it sounds like without this lesson.}
\end{figure}
\end{frame}

\section{Conclusion}

\begin{frame}{Summary}
\begin{enumerate}
\item \textbf{Week 1}: \textit{Concise writing I:} In Week 1, we will focus on creating concise writing that eliminates extraneous words and sentences from your writing.
\begin{itemize}
\item Exercises: distilling complex scientific articles into shorter tracts of writing
\item Exercises: reading popular scientific journal articles and discussing them in small groups
\item Exercises: Practice using analogies in communicating difficult or abstract scientific thoughts
\item Homeworks: practice descriptive detail, placing it into a hierachy that makes sense
\item Exploration topic: gravitational waves
\end{itemize}
\end{enumerate}
\end{frame}

\bibliographystyle{plain}
\section{Bibliography}

\bibliography{bibfile}

\end{document}
