\documentclass{article}
\usepackage{graphicx}
\usepackage[margin=1.5cm]{geometry}
\usepackage{amsmath}

\begin{document}

\title{Week 2 Writing Activity: \textit{Concise Writing and the Demarcation Problem}}
\author{Prof. Jordan C. Hanson (INTD100)}

\maketitle

\section{Chapter 1: Scientific Method and the Problem of Demarcation}

\begin{enumerate}
\item In chapter 1 of our course book, we encounter the following:
\begin{quotation}
Sociologists of science, relativists, postmodernists, social constructivists, and others took their turns attacking the idea that science was rational, that it had anything to do with the pursuit of truth, or indeed that scientific theories were anything more than a reflection of the political biases about race, class, and gender held by the scientists who produced them. To some, science became an ideology, and facts and evidence were no longer automatically accepted as providing credible grounds for theory choice even in the natural sciences.
\end{quotation}
Write a paragraph about the context of the situation, in a concise form.  Recall how philosophers like Larry Laudan had asserted that there is no solution to the demarcation problem.  How does science proceed in spite of the above critiques? \\ \vspace{3cm}
\item Is \textit{creationism} science?  Why or why not?  If it is not, explain why as concisely as you can.  If it is, explain what evidence could falsify it.
\\
\vspace{2cm}
\end{enumerate}

\section{Technical Writing Exercise 3: Concise Writing}

Create a short paragraph that defines the following set of ideas and links them together: \textit{scientific}, \textit{non-scientific}, \textit{pseudo-scientific}, \textit{unscientific}.  Give a subject example for each.  Consider Figure 1.1 of the book.


\end{document}
