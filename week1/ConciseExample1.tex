\documentclass{article}
\usepackage{graphicx}
\usepackage[margin=1.5cm]{geometry}
\usepackage{amsmath}

\begin{document}

\title{Week 1 Writing Activity: \textit{Concise Writing}}
\author{Prof. Jordan C. Hanson (INTD100)}

\maketitle

\section{Technical Writing Exercise 1: Instructions for a Recipe}

Face the refrigerator in the kitchen.  Open the left door, and take the palette of eggs.  Place them on the wooden cutting board to the left of the refrigerator.  Remove seven eggs and leave them on the board.  Replace the palette in the refrigerator.  Lift the jug of olive oil from behind the board.  Pour oil into the blue frying pan on the stove to the left of the board until a circle with radius 3 cm has formed.  Replace the jug, and tear a paper towel from the roll to the right of the cutting board.  Use the paper towel to spread the oil evenly over the pan surface.  Ignite the flame under the pan using the switch on the upper right side of the stove top.  Directly across the kitchen, above the sink, are plastic containers.  Take one, along with a ceramic bowl from the cupboard above and left of the sink.  Place the container and the bowl on the cutting board.  Crack each egg on the board, dropping the contents into the bowl while discarding the shell in the container.  In the drawer to the right of the sink, there are forks.  Take a rinsed fork and whisk the eggs until an even slurry has formed.  In the cupboard above the board, there are containers of salt and pepper.  Sprinkle salt and pepper into the egg slurry as desired.  When a light smoke begins to drift from the hot pan, pour the slurry into the pan.  Place the bowl and container in the sink, and grab a spatula from the metal cylinder behind the board.  After a solid edge of cooked egg has formed around the perimeter of the pan, use the spatula to mix the omlette until a solid mass has formed.  Turn off the flame.  Take a plate from the stack to the right of the sink, and transfer the omlette to the plate.

\end{document}
