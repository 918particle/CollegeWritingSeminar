\documentclass{article}
\usepackage{graphicx}
\usepackage[margin=1.5cm]{geometry}
\usepackage{amsmath}

\begin{document}

\title{Week 2 Writing Activity: \textit{Concise Writing and Misconceptions about Science}}
\author{Prof. Jordan C. Hanson (INTD100)}

\maketitle

\section{Chapters 1-2 of \textit{The Scientific Attitude}: Dermarcation and Misconceptions about Science}

\begin{enumerate}
\item From the reading, discuss the issues of \textit{confirmation} and \textit{falsification} as they pertain to the orbits of the planets Mercury, Saturn, and Neptune.  Was Isaac Newton's theory of gravity rejected by observations of the orbit of Saturn?  How did general relativity finally explain the orbit of Mercury? \\ \vspace{3cm}
\item For each subject below, determine if it is \textit{scientific}, \textit{pseudo-scientific}, or \textit{non-scientific}. (a) Political science, (b) Psychology, (c) Flat Earth theory, (d) Chemistry, (e) Mathematics, (f) Philosophy, (g) Astrology, (h) Economics.\\ \vspace{1cm}
\end{enumerate}

\section{Technical Writing Exercise 4: Concise Writing}

\begin{enumerate}
\item Make the following tract of writing more concise:
\begin{quotation}
Our program that we have created helps the students to learn physics by giving them questions.  The questions are multiple choice.  The multiple choice questions are conceptual problems presented in a way that reveals if the student really gets the concept.  The questions are drawn from a memory bank and the questions the students usually get right are gradually removed from the bank until the student has mastered them.
\end{quotation} \vspace{2cm}
\item Imagine hiding a tool in a large laboratory.  Now give someone a set of instructions that shows them precisely where to find it.
\end{enumerate}


\end{document}
