\documentclass{article}
\usepackage{graphicx}
\usepackage[margin=1.5cm]{geometry}
\usepackage{amsmath}

\begin{document}

\title{Week 2 Writing Activity: \textit{Concise Writing and the Demarcation Problem}}
\author{Prof. Jordan C. Hanson (INTD100)}

\maketitle

\section{Chapter 1: Scientific Method and the Problem of Demarcation}

\begin{enumerate}
\item In chapter 1 of our course book, we encounter a critique of the \textit{scientific method.}  Define, in your own words, the scientific method and how it is applied.
\\
\vspace{2cm}
\item Reflect on the logic of the following statements.  Why or why not do they make sense?  Think of an example or a counter-example.
\begin{itemize}
\item \textit{My hypothesis states that a neutron will always decay into a proton and an electron.  My observations confirm that I observe a proton and an electron for each neutron decay.  My hypothesis is confirmed.}: \\ \vspace{2.5cm}
\item \textit{My hypothesis states that a neutron will always decay into a proton and an electron.  I continue to make observations of neutron decays, and when and if I see a neutron decay into a combination of particles other than a proton and an electron, I will reject my hypothesis.}: \\ \vspace{2.5cm}
\end{itemize}
\end{enumerate}

\section{Technical Writing Exercise 2: Distillation}

\begin{quotation}
``If someone was born between 1945 and 1991, then they have Strontium-90 in their bones.  Eve has Strontium-90 in her bones.  Therefore, Eve was born between 1945 and 1991.'' Obviously, this kind of argument is not deductively valid.  The fact that Eve has Strontium-90 in her bones is no guarantee that she was born between 1945 and 1991.  Eve might, for example, have grown up near a nuclear reactor in Pennsylvania in the late 1990s, where it was found that Strontium-90 was present as a result of environmental contamination.  \textbf{Distill this paragraph into 1-2 sentences, maximum.}

\end{quotation}

\end{document}
