\documentclass{article}
\usepackage{graphicx}
\usepackage[margin=1.5cm]{geometry}
\usepackage{amsmath}
\usepackage{hyperref}

\begin{document}

\title{Week 11 Writing Activity: Writing Mechanics in Essays and Articles and Chapter 8 of \textit{The Scientific Attitude}}
\author{Prof. Jordan C. Hanson (INTD100)}

\maketitle

\section{Pseudo-Science and Denialism}

\begin{enumerate}
\item In your own words, give definitions of \textit{denialism} and \textit{pseudo-science}. \\ \vspace{1cm}
\item \textit{(Personal reflection).} Write down a belief or set of beliefs you hold that would be difficult to drop if scientific evidence was presented showing these to be false.  Are you actually prepared to drop them?  What can be said about your belief if you \textit{are} willing to drop them, but have not yet? \\ \vspace{1.5cm}
\item Given what we have written about leadership, how would you approach someone who believes something that is demonstrably false? \\ \vspace{1.5cm}
\end{enumerate}

\section{Splitting the Sentence}

A common mistake in essays and articles is to create sentences that contain too many ideas.  Below are examples of sentences that need to be split into several sentences, with one idea per sentences.  Split each sentence into \textit{at least two} sentences.

\begin{enumerate}
\item Once I become a chemical engineer, I would like to give back to my community by creating small-scale nuclear reactors that help power small towns because I believe this will give them a more affordable and higher quality of life and will help address climate change. \\ \vspace{1cm}
\end{enumerate}

\section{Creating an Abstract}

Imagine an experiment where you measure the density of an object my submerging it in water, and then weighing it.  Write an abstract describing the results.  You would first measure the volume of an object by submerging it and measuring the increase in water level.  Second, you would form a ratio of the mass divided by the volume to give the density.

\end{document}
