\title{Syllabus for College Writing Seminar (Hanson): INTD100 Fall Module 1}
\author{Dr. Jordan Hanson - Whittier College Dept. of Physics and Astronomy}
\date{\today}
\documentclass[10pt]{article}
\usepackage[a4paper, total={18cm, 27cm}]{geometry}
\usepackage{hyperref}
\begin{document}
\maketitle

\begin{abstract}
\textbf{Coffee and Black Holes:} \textit{An Introduction to Science Writing.}  Science writing is useful, and we encounter it in a variety of styles that share common elements. Good science writing may be found in popular science sites and magazines which distill scientific breakthroughs into accessible language for a general audience. Popular science writing involves metaphor, analogies, descriptive detail, and quotation and paraphrasing of experts. Another example of science writing may be found in technical descriptions, which must convey an idea accurately enough so that another person may understand the shape, form, or design of something without any mistake. Technical description requires practice in the art of eliminating ambiguity from writing. Finally, science writing takes the form of thesis, dissertation, and journal article writing, in which scientific results are reported. Students will regularly practice these styles in a variety of in-class writing exercises and assignments.  Additionally, exciting topical subject matter will be explored, including the discovery of gravitational waves, black hole observations, vaccine use, COVID-19, and climate science.
\end{abstract}
\noindent
\textit{\textbf{Pre-requisites}: None} \\
\textit{\textbf{Course announcements}: There will be regular course announcements via the content management system (CMS) of Whittier College, Moodle.  Moodle emails and messages should appear at your Whittier College email address.} \\
\textit{\textbf{Course credits, Liberal Arts Categorization}: 3 Credits, INTD100 is a required course.} \\
\textit{\textbf{Regular course hours}: Monday through Friday, 8:00-9:15.  Synchronous time: 30 minutes to 1 hour.  ``Synchronous'' time corresponds to meeting together via Zoom. ``Asynchronous'' time corresponds to engaging with pre-recorded lectures, and performing writing exercises due the next day.} \\
\textit{\textbf{Instructor contact information}: jhanson2@whittier.edu, Zoom ID: 796 092 0745, 667725} \\
\textit{\textbf{Office hours}: Each day after class for the 1 hour following class.} \\
\textit{\textbf{Attendance/Absence}: Student attendance will be taken each class period.  Students are allowed three absences at no penalty, no questions asked.  For further absences, consult with the instructor.  If a student has more than three absences and no consultation with the instructor, the grading penalty is 1.5 \% per class period missed}.\\ 
\textit{\textbf{Late work policy}: Late work is generally not accepted, but consult the instructor in special circumstances.} \\
\textit{\textbf{One-on-one discussions:} Students will be required to meet for 15-30 minutes one-on-one with the instructor to review and enhance writing tracts.  Details will be arranged in Week 1.} \\
\textit{\textbf{Textbook}: The only required book for the course is \textit{\textbf{Notes of a Native Son} by James Baldwin}.  The summer assignment comes from one essay in this book, but it is a fascinating collection of essays on the social dynamics of the United States in the mid-20th century up until today.} \\
\textit{\textbf{Grading}: There will be one midterm, one final exam, and daily writing exercises.  There will also be homework in the form of writing assignments.  Usually assignments will be shared with the instructor via students' Google Drive.  Daily assignments are graded for completion, while homework and exams are assessed for points\footnote{This prevents a student losing points for having writer's block, for example, or just an off-day.  See Tab. \ref{tab:grades} for the grading weights of assignments.}} \\

\begin{table}[h]
\centering
\begin{tabular}{| c | c |}
\hline
Item & Percentage \\ \hline \hline
Daily exercises and attendance & 20 \% \\ \hline
Writing assignments: homework & 30 \% \\ \hline
Midterm exam & 20 \% \\ \hline
Final exam & 15 \% \\ \hline
Final writing assignment & 15\% \\ \hline
\end{tabular}
\begin{tabular}{| c | c |}
\hline
Item & Percentage \\ \hline \hline
Daily exercises and attendance & 25 \% \\ \hline
Writing assignments: homework & 35 \% \\ \hline
Midterm exam & 20 \% \\ \hline
Final writing assignment & 20\% \\ \hline
\end{tabular}
\caption{\label{tab:grades} (Left) These are the grading weights for each assignment class.  For example, 30 percent of the grade will come from the writing assignments from homework.  If a student earns 5/6 of these points, then the final grade will gain only 25 percent (rather than the full 30 percent). (Right) If a student decide to opt-out of the final exam, the weights on the right will be used instead.}
\end{table}
\textit{\textbf{Grade Settings}: $\geq 60\%, <70\%$ = D, $\geq 70\%, <80\%$ = C, $\geq 80\%, <90\%$ = B, $\geq 90\%, <100\%$ = A.  \\ Pluses and minuses: 0-3\% minus, 3\%-6\% straight, 6\%-10\% plus (e.g. 79\% = C+, 91\% = A-)} \\ \\
\textit{\textbf{ADA Statement on Disability Services}: The Americans with Disabilities Act (ADA) is a federal anti-discrimination statute that provides comprehensive civil rights protection for persons with disabilities. Among other things, this legislation requires that all students with disabilities be guaranteed a learning environment that provides for reasonable accommodation of their disabilities. If you believe you have a disability requiring an accommodation, please contact Disability Services: disabilityservices@whittier.edu, tel. 562.907.4825.} \\
\textit{\textbf{Academic Honesty Policy}: \url{http://www.whittier.edu/academics/academichonesty}} \\

\textit{\textbf{Policy due to COVID-19}: Our course is an online course, so there is no additional risk of transmission from person-to-person contact.  However, we do note some policies that should be helpful:
\begin{enumerate}
\item Class will meet \textit{synchronously} for \textbf{up to 1 hour} using the Zoom web-conferencing application, at the normal class time.  It is necessary to download the Zoom application to your preferred device, and secure a quiet location where you will be able to hear and see the instructor.  Please see \url{https://zoom.us/download} for access to Zoom.  We will reach consensus in the first days of class as to the exact start time of the daily synchronous portion of the course.
\item \textbf{Occasionally, students will be required to present to a small group or the whole class.}  In such cases, one must activate screen sharing so the group or class can see and hear the presenter.  Thus, it is important the student has a suitable study area.  If this is not feasible, make sure to contant the instructor as early as possible to make arrangements.
\item \textbf{Students may opt-out of taking the final exam.}  This policy is meant to accommodate students' wide variety of learning situations during the pandemic.  If one does not take it, the assignment weights for the final grade will be those given in Tab. \ref{tab:grades} (right).
\end{enumerate}}

\textit{\textbf{Course Objectives}:}
\begin{itemize}
\item To practice written and oral expression of technical ideas.
\item To perform daily exercises that will strengthen the clarity and conciseness of technical writing.
\item To understand and create science writing that distills complex ideas for a general audience.
\item To practice precise technical description, or writing free of ambiguity.
\item To receive training in refining, clarifying, and organizing the flow of scientific information in a report.
\end{itemize}
\noindent\rule{18cm}{0.4pt}\\
\textit{\textbf{Summer Reading Assignment:} Please acquire Notes of a Native Son by James Balwin, and read at least ``Notes of a Native Son,'' the essay within this collection of essays.  Students are assigned the following, which is due August 31st.}
\begin{enumerate}
\item Create a Google Document in your shared Whittier College Google Drive.  All first-year students should be given a Google Drive shared space by Whittier College.  Please contact IT services with any issues (\url{https://www.whittier.edu/it/contact}).
\item Create a enumerated list (1., 2., 3. etc.).  Each item is to a separate mini-assignment, and each could take between a few minutes and an hour.
\item Item 1: This essay is in a narrative style, and is broken into several parts.  Summarize part I in one paragraph, using 120 words or fewer.  \textit{Note:} The purpose of this exercise is to complete the summary in \textit{no more} than a certain number of words.  What key moments stand out in your mind?  What is the central realization of the author at the end of the section?
\item Item 2: Repeat exercise 1, but instead use \textit{only twenty words.}  What happens in part I of the essay?
\item Item 3: Consider part III.  Construct a tract of writing that a) defines and explains the \textit{poison} metaphor the author describes at the funeral of his father, b) identifies the author's ``cure'' or cures for the poison, and c) provides several pieces of supporting evidence for the identification of the author's cure.
\item Item 4: Consider the final part of the essay, when the author describes the fight in the Hotel Braddock.  Write a tract of between 200-400 words on the author's treatment of \textit{evidence and facts.}  What does the author have to say about the importance of facts about the fight to the people in that neighborbood?  Notice the author's writing takes on the tone of a reporter regarding the ensuing riot.  What facts stand out regarding the outcome?
\item Item 5: \textbf{Bonus points:} Read the essay ``The Harlem Ghetto'' from earlier in the book.  How do the details in this essay corroborate the sentiments of the characters in the first essay?  This item can be any number of words.
\end{enumerate}
\noindent\rule{18cm}{0.4pt} \\ \\
\clearpage
\textit{\textbf{Module 1 Outline}:}
\begin{enumerate}
\item \textbf{Week 1}: \textit{Concise writing I:} In Week 1, we will focus on creating concise writing that eliminates extraneous words and sentences from your writing.
\begin{itemize}
\item Exercises: distilling complex scientific articles into shorter tracts of writing.
\item Exercises: reading popular scientific journal articles and discussing them in small groups
\item Exercises: Practice using analogies in communicating difficult or abstract scientific thoughts
\item Homeworks: Practice quotation and paraphrasing of experts in writing
\item Homeworks: practice descriptive detail by providing the reader with the correct details such that they understand something complex
\item Exploration topic: gravitational waves
\end{itemize}
\item \textbf{Week 2}: \textit{Concise writing II:} In Week 2, we will focus on reading a piece of science writing, and creating your \textit{own} writing that tailors the story to a particular audience.
\begin{itemize}
\item Exercises: work in teams to produce a piece of science writing intended for a broad audience that weaves together information from a variety of sources: (a) a TED talk (b) a scientific journal article (c) and resources like Wikipedia and Google Scholar
\item Homework: writing a post designed for social media about a piece of science that grabs the attention of a wide audience, and attempts to convince that audience that the science is interesting
\item Exploration topic: Black hole observations
\end{itemize}
\item \textbf{Week 3}: \textit{Technical description I:} In Week 3, we will focus on removing ambiguity from descriptions of a wide variety of situations including (a) scenes or settings, (b) drawings or diagrams, and (c) locations on the map or in a space.
\begin{itemize}
\item Exercises: Working in pairs, describe the location of an object in a photograph without revealing what it is to the other, then writing that down
\item Exercises: Working in pairs, describe a technical diagram with the goal of developing instructions for assembly
\item Homework: write an unambiguous set of instructions for executing a task like the performance of a scientific experiment, procedure, or calculation
\item Exploration topic: COVID-19 and pandemics
\end{itemize}
\item \textbf{Week 4}: \textit{Technical description II:} We will focus in Week 4 on the passage of time, and describing technically procedures in the passive voice.
\begin{itemize}
\item Exercises: write about a simple experiment you perform at home with household items in the present tense, and then change the writing to passive voice
\item Exercises: write about a simple experiment you perform at home with household items, and practice removing the subject pronouns.  For example, replacing ``I observed three birds outside'' with ``Three birds were observed.''
\item Homeworks: describe a personal acheivement you experienced in high school, but include ample usage of passive voice and technical descriptive language.  Would someone be able to repeat what you did?
\item Homeworks: trade tracts of your technical descriptive writing with a classmate, and you each perform the procedure or process described and write about the outcome.  Compare notes with your partner and identify inaccuracies.
\item \textbf{Midterm}: The midterm will be due at the end of Week 4, involving a take-home style exam testing what concepts the student has gained up to this point.
\item Exploration topic: Vaccines
\end{itemize}
\item \textbf{Week 5}: \textit{The Laboratory Report, I} We will focus on the ubiquitous topic of lab report construction in Week 5.
\begin{itemize}
\item Exercises: how to include a graphic, figure, or table into your writing, and write about it correctly
\item Exercises: how to write an abstract
\item Exercises: how to organize a report into sections and sub-sections
\item Homework: find a data set, any data set, and use it to construct a lab report
\item Exploration topic: Climate science and climate science skepticism
\end{itemize}
\item \textbf{Week 6}: \textit{The Laboratory Report, II} We continue in Week 6 with lab reports, focusing on polishing them and using citations
\begin{itemize}
\item Exercises: practicing creating citations in different formats
\item Exercises: determining when a citation is required
\item Exercises: \textbf{Important}: Using the lab report created in the prior week, and other examples, we will practice working with the thread of logic and thought that runs through any academic article or piece.  We will focus on holding the reader's attention, and leading the reader from one concept to another such that he or she understands each step of the writer's ideas.
\item Exploration topic: Climate science and climate science skepticism
\end{itemize}
\item \textbf{Week 7}: Course review and preparation for the final exam
\begin{itemize}
\item \textbf{Final writing assignment}.  Create a 1000 word piece that (a) summarizes an interesting or ground-breaking piece of science that you encountered in your reading in Module I, (b) includes passive voice, technical descriptive language, (c) and the proper use of analogy and metaphor to communicate the science to an audience of your peers.  This writing should be single-spaced, with figures and references.
\item The final exam is optional, and will be the same format as the midterm.  It will cover all seven weeks, however.
\item Exploration topic: Coffee production in Latin America and Southern California, and the work of one Whittier College Professor \url{https://cinziafissore.com/}
\end{itemize}
\end{enumerate}
\end{document}
