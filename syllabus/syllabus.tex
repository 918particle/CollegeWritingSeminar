\title{Syllabus for College Writing Seminar (Hanson): INTD100 Fall Module 1}
\author{Dr. Jordan Hanson - Whittier College Dept. of Physics and Astronomy}
\date{\today}
\documentclass[10pt]{article}
\usepackage[a4paper, total={18cm, 27cm}]{geometry}
\usepackage{hyperref}
\begin{document}
\maketitle

\begin{abstract}
\textbf{The Scientific Attitude:} \textit{An Introduction to Science Writing.}  Science writing is a useful skill.  We encounter it in a variety of styles that share common elements. Good science writing may be found in popular science sites and magazines which distill scientific breakthroughs for a general audience.  This writing involves metaphor, analogies, descriptive detail, and quotation and paraphrasing of experts.  It is also found in technical descriptions, which must convey an idea such that another person may understand the shape, form, or design of something without ambiguity.  Finally, science writing takes the form of thesis, lab reports, and journal article writing, to convey results.  Though science often appears objective, contemporary movements seek to undermine science, or question its validity, often through writing that appears scientific.  Students will regularly practice scientific writing in a variety of in-class writing exercises and assignments.  Students will also dive into reading that articulates and defends the ideas and structure of science.  Students will discuss what sets scientific thinking and writing apart from other forms of thought, with the ultimate goal of developing a \textit{scientific attitude} that can be used to ground one's ideas in the structures and language of science. 
\end{abstract}
\noindent
\textit{\textbf{Pre-requisites}: None} \\
\textit{\textbf{Course announcements}: Students will be invited to join the class Discord server: \url{https://discord.gg/GqUkQSxf}.} \\
\textit{\textbf{Course credits, Liberal Arts Categorization}: 3 Credits, INTD100 is a required course.} \\
\textit{\textbf{Regular course hours}: Monday, Wednesday, and Friday, 9:00-9:50, SLC 104.} \\
\textit{\textbf{Instructor contact information}: jhanson2@whittier.edu, Zoom ID: 796 092 0745, 667725.  Discord: 918particle\#5083} \\
\textit{\textbf{Booking online office hours:} Please contact instructor via Discord for video chat.} \\
\textit{\textbf{Attendance/Absence}: Instructor discretion.  Attendence will be taken during each class period, but absences due to family emergencies, illness or mental health, athletics, are all acceptable with no questions asked.}\\ 
\textit{\textbf{Late work policy}: Late work is generally not accepted, but consult the instructor in special circumstances.} \\
\textit{\textbf{Textbook}: The only required book for the course is \textit{\textbf{The Scientific Attitude} by Lee McIntyre}.  We will pace reading such that the book is covered by the end of the semester.} \\
\textit{\textbf{Grading}: There will be one midterm, daily writing exercises, one final writing project, and homework in the form of writing assignments.  Usually assignments will be shared with the instructor via Discord and stored on Moodle.  Daily assignments are graded for completion, while homework and exams are assessed for points\footnote{This prevents a student losing points for having writer's block, for example, or just an off-day.  See Tab. \ref{tab:grades} for the grading weights of assignments.}}

\begin{table}[h]
\centering
\begin{tabular}{| c | c |}
\hline
Item & Percentage \\ \hline \hline
Daily exercises and attendance & 20 \% \\ \hline
Writing assignments: homework & 30 \% \\ \hline
Midterm exam & 20 \% \\ \hline
Final writing project & 30\% \\ \hline
\end{tabular}
\caption{\label{tab:grades} These are the grading weights for each assignment class.  For example, 30 percent of the grade will come from the writing assignments from homework.  If a student earns 5/6 of these points, then the final grade will gain only 25 percent (rather than the full 30 percent).}
\end{table}
\textit{\textbf{Grade Settings}: $\geq 60\%, <70\%$ = D, $\geq 70\%, <80\%$ = C, $\geq 80\%, <90\%$ = B, $\geq 90\%, <100\%$ = A.  \\ Pluses and minuses: 0-3\% minus, 3\%-6\% straight, 6\%-10\% plus (e.g. 79\% = C+, 91\% = A-)} \\ \\
\textit{\textbf{Student Accessibility Services}: Whittier College is committed to make learning experiences as accessible as possible. If you experience physical or academic barriers due to a disability, you are encouraged to contact Student Accessibility Services (SAS) to discuss options. To learn more about academic accommodations, drop by our center (ground floor of Wardman Library), email sas@whittier.edu, or contact 562-907-4825.} \\
\textit{\textbf{Academic Honesty Policy}: \url{http://www.whittier.edu/academics/academichonesty}}
\clearpage
\textit{\textbf{Course Learning Goals}:}
\begin{itemize}
\item To practice written and oral expression of technical ideas.
\item To perform daily exercises that will strengthen the clarity and conciseness of technical writing.
\item To understand and create science writing that distills complex ideas for a general audience.
\item To practice precise technical description, or writing free of ambiguity.
\item To receive training in refining, clarifying, and organizing the flow of scientific information in a report.
\end{itemize}

\textit{\textbf{Course Outline}:}
\begin{enumerate}
\item \textbf{Unit 1}: \textit{Concise writing I:} In Week 1, we will focus on creating concise writing that eliminates extraneous words and sentences from your writing.
\begin{itemize}
\item Exercises: distilling complex scientific articles into shorter tracts of writing
\item Exercises: reading popular scientific journal articles and discussing them in small groups
\item Exercises: Practice using analogies in communicating difficult or abstract scientific thoughts
\item Homeworks: Practice quotation and paraphrasing of experts in writing
\item Homeworks: practice descriptive detail by providing the reader with the correct details such that they understand something complex
\item Exploration topic/reading: Chapters 1 and 2 of course book.  \textit{Demarcation of science, misconceptions about science.}
\end{itemize}
\item \textbf{Unit 2}: \textit{Concise writing II:} In Unit 2, we will focus on reading a piece of science writing, and creating your \textit{own} writing that tailors the story to a particular audience.
\begin{itemize}
\item Exercises: work in teams to produce a piece of science writing intended for a broad audience that weaves together information from a variety of sources: (a) a TED talk (b) a scientific journal article (c) and resources like Wikipedia and Google Scholar
\item Homework: writing a post designed for social media about a piece of science that grabs the attention of a wide audience, and attempts to convince that audience that the science is interesting
\item Exploration topic/reading: Chapters 3 and 4 of course book. \textit{The scientific attitude and demarcation}
\end{itemize}
\item \textbf{Unit 3}: \textit{Technical description I:} In Unit 3, we will focus on removing ambiguity from descriptions of a wide variety of situations including (a) scenes or settings, (b) drawings or diagrams, and (c) locations on the map or in a space.
\begin{itemize}
\item Exercises: Working in pairs, describe the location of an object in a photograph without revealing what it is to the other, then writing that down
\item Exercises: Working in pairs, describe a technical diagram with the goal of developing instructions for assembly
\item Homework: write an unambiguous set of instructions for executing a task like the performance of a scientific experiment, procedure, or calculation
\item Exploration topic/reading: Chapters 5 and 6 of course book. \textit{Embracing the scientific attitude practically, and transformation of modern medicine}
\end{itemize}
\item \textbf{Unit 4}: \textit{Technical description II:} We will focus in Unit 4 on the passage of time, and describing technically procedures in the passive voice.
\begin{itemize}
\item Exercises: write about a simple experiment you perform at home with household items in the present tense, and then change the writing to passive voice
\item Exercises: write about a simple experiment you perform at home with household items, and practice removing the subject pronouns.  For example, replacing ``I observed three birds outside'' with ``Three birds were observed.''
\item Homeworks: describe a personal acheivement you experienced in high school, but include ample usage of passive voice and technical descriptive language.  Would someone be able to repeat what you did?
\item Homeworks: trade tracts of your technical descriptive writing with a classmate, and you each perform the procedure or process described and write about the outcome.  Compare notes with your partner and identify inaccuracies.
\item \textbf{Midterm}: The midterm will be due at the end of Unit 4, involving a take-home style exam testing what concepts the student has gained up to this point.
\item Exploration topic/reading: Chapters 7 and 8 of course book. \textit{Science gone wrong, science gone sideways}
\end{itemize}
\item \textbf{Unit 5}: \textit{The Laboratory Report, I} We will focus on the ubiquitous topic of lab report construction in Unit 5.
\begin{itemize}
\item Exercises: how to include a graphic, figure, or table into your writing, and write about it correctly
\item Exercises: how to write an abstract
\item Exercises: how to organize a report into sections and sub-sections
\item Homework: find a data set, any data set, and use it to construct a lab report
\item Exploration topic/reading: Chapters 9 and 10 of course book.  \textit{The case for social sciences, valuing science in general}
\end{itemize}
\item \textbf{Unit 6}: \textit{The Laboratory Report, II} We continue in Unit 6 with lab reports, focusing on polishing them and using citations
\begin{itemize}
\item Exercises: practicing creating citations in different formats
\item Exercises: determining when a citation is required
\item Exercises: \textbf{Important}: Using the lab report created in the prior Unit, and other examples, we will practice working with the thread of logic and thought that runs through any academic article or piece.  We will focus on holding the reader's attention, and leading the reader from one concept to another such that he or she understands each step of the writer's ideas.
\item Exploration topic: choosing topics for final writing projects
\end{itemize}
\item \textbf{Unit 7}: Course review and preparation for the final writing project
\begin{itemize}
\item \textbf{Final writing project: option A}.  Create a 1000 word piece that (a) summarizes an interesting or ground-breaking piece of science that you encountered in your reading in in the course, (b) includes passive voice, technical descriptive language, (c) and the proper use of analogy and metaphor to communicate the science to an audience of your peers.  This writing should be single-spaced, with figures and references.
\item \textbf{Final writing project: option B}.  Create a 1000 word piece that (a) describes in lab-report format how you designed, built, and conducted a scientific experiment or statistical analysis, (b) includes passive voice, technical descriptive language, (c) and the proper use of analogy and metaphor to communicate the science to an audience of your peers.  This writing should be single-spaced, with figures and references.
\end{itemize}
\end{enumerate}
\end{document}
