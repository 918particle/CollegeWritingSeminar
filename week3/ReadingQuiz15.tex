\documentclass{article}
\usepackage{graphicx}
\usepackage[margin=1.5cm]{geometry}
\usepackage{amsmath}
\usepackage{hyperref}

\begin{document}

\title{Week 6 Writing Activity: Chapter 5 of the \textit{Scientific Attitude}}
\author{Prof. Jordan C. Hanson (INTD100)}

\maketitle

\section{Practical Ways in Which Scientists Embrace the Scientific Attitude}

\begin{enumerate}
\item We will now shamelessly hack some p-values.  Follow this link: \url{https://projects.fivethirtyeight.com/p-hacking/}.  (1) Choose one of two major political parties in the United States at the top.  We are asking the question: ``Does the economy do better if politicians from this or that party is in power?'' (2) Define what you mean by politicians on the left side.  (3) Run the analysis, given options like the amount of power the politicians have (US Representative, President, etc.) and whether or not recessions are included.  Note the linear regression is used to produce a correlation. (4) Check if your result is significant on the right side.  Do you feel comfortable publishing these results? \\ \vspace{1cm}
\item Consider the following quote from Ch. 5 of the text:
\begin{quote}
It should be clear by now how the conclusions of Settle, Koertge, and Longino tie in neatly with Sunstein's work.  It is not hust the honesty or ``good faith'' of the individual scientist, but fidelity to the scientific attitude as community practice that makes science special as an institution.  No matter the biases, beliefs, or petty agendas that may be put forward by individual scientists, science is more objective than the sum of its individual practitioners.
\end{quote}
\begin{itemize}
\item Recall that Cass Sunstein's work is about the wisdom of crowds.  Tom Settle is the author who articulates the \textit{actual} difference between the warrant we place in science and magic.  Noretta Koertge and Helen Longino both argue that the eventual objectivity of science arises out of the critical communities who actively engage in eliminating error from scientific results.
\item What are the characteristics mentioned in the text, or in your own opinion, that make for a healthy critical community? \\ \vspace{3cm}
\item How would you feel about dissenting within your social group if you felt that it was warranted?  What qualities are important in a person who can do this?
\end{itemize}
\end{enumerate}

\end{document}
