\documentclass{article}
\usepackage{graphicx}
\usepackage[margin=1.5cm]{geometry}
\usepackage{amsmath}
\usepackage{hyperref}

\begin{document}

\title{Week 7 Writing Activity: Chapter 5 of the \textit{Scientific Attitude}}
\author{Prof. Jordan C. Hanson (INTD100)}

\maketitle

\section{Technical Writing - Peer Review, Data Sharing, and Replication}

\textit{Below you will find tracts of writing from a study on crime statistics\footnote{J.C. Hanson \textit{et al}.  ``Social Factor Analysis of Fluctuating Crime Rates in Major US Cities (1991-2022): Education, Wealth Inequality, and Economic Growth.'' \textit{Journal of Totally not True Crime Statistics} vol. 99 n. 2, (2048).}.  Your job is to act as a peer reviewer.  Working in pairs or solo, your choice, write your critique on the back page after reading the passages.} \\ \\
\small
\textbf{Abstract}\\
Though crime rates in the United States have generally declined since the early 1990s, new trends indicating this four decade decline is coming to an end.  Rates of violent crime have begun to increase in seven cities we selected from the list of all cities in the United States with populations above 500,000.  In our definition of violent crime, we include crimes related to property damage, theft, burglary, breaking and entering, accidental death, manslaughter, and murder.  Data was collected from the FBI Uniform Crime Reporting Program (UCR), for the specific years of 1991, 2018, and 2022.  The average rate for the violent crime categories we list show increases over this time span.  Though there are many potential causes, we discuss educational trends, growth and productivity, and wealth inequality as potential contributors. \\ \\
... \\
\begin{table}[ht]
\centering
\begin{tabular}{| c | c | c | c |} \hline
US City & Rate (1991) & Rate (2018) & Rate (2022) \\ \hline
City 1 & 12\% & 12\% & 18\% \\ \hline
City 2 & 4\% & 5\% & 18\% \\ \hline
City 3 & 20\% & 16\% & 18\% \\ \hline
City 4 & 3\% & 4\% & 18\% \\ \hline
City 5 & 6\% & 9\% & 18\% \\ \hline
City 6 & 14\% & 18\% & 18\% \\ \hline
City 7 & 8\% & 14\% & 18\% \\ \hline
\end{tabular}
\caption{\label{tab:1} Rates of theft for selected cities for the years 1991, 2018, and 2022.}
\end{table}

The rates of theft for selected cities are shown in Tab. 2.3.  The relative increase in rates is observed for all categories, including by city and over time.
\\
... \\
\begin{table}[ht]
\centering
\begin{tabular}{| c | c | c | c |} \hline
US City & Rate (1991) & Rate (2018) & Rate (2022) \\ \hline
City 1 & 4\% & 8\% & 11\% \\ \hline
City 2 & 1\% & 5\% & 12\% \\ \hline
City 3 & 7\% & 14\% & 9\% \\ \hline
City 4 & 1\% & 3\% & 17\% \\ \hline
City 5 & 3\% & 10\% & 8\% \\ \hline
City 6 & 4\% & 7\% & 15\% \\ \hline
City 7 & 2\% & 5\% & 12\% \\ \hline
\end{tabular}
\caption{\label{tab:2} Rates of murder for selected cities for the years 1991, 2018, and 2022.}
\end{table}

The rates of murder for selected cities are shown in Tab. 3.1.  We will show in Sec. 4.4 that the numbers are correlated with wealth inequality and educational opportunities ($p \leq 0.024$).  The results not are not strongly correlated with economic growth ($p \geq 0.05$).

\end{document}
