\documentclass{article}
\usepackage{graphicx}
\usepackage[margin=1.5cm]{geometry}
\usepackage{amsmath}
\usepackage{hyperref}

\begin{document}

\title{Week 6 Writing Activity: Technical Description I}
\author{Prof. Jordan C. Hanson (INTD100)}

\maketitle

\section{Technical Description I}

\textit{Many people think they can describe something.} - Have you ever been in a situation in which you were \textit{sure} someone understood you, only to find out they had the wrong idea?  What is required of good technical description?
\begin{enumerate}
\item \textbf{Specific details}
\begin{itemize}
\item Spatial: location in 2D/3D space, geographic
\item Temporal: order, past or future
\item Observable: shape, color, structure, orientation
\end{itemize}
\item \textbf{Objective details}
\begin{itemize}
\item Which way is up? Discussions of physical perspective
\item Dispassionate description
\item Removal of ambiguous words
\end{itemize}
\end{enumerate}
\textbf{Why is this necessary to study and practice?} Science, which relies on the scientific method, is impossible without \textit{repeatability}.

\begin{enumerate}
\item Think of an experiment in physics, chemistry, or biology that you understand well.  Write below a set of procedures that, if read by a stranger, would like someone to produce the results you expect. \\ \vspace{7cm}
\item When time is called, trade papers with a partner and read their tract.  Highlight any phrases or sentences that feel ambiguous, meaning you would not know precisely what to do next.
\end{enumerate}

\end{document}
