\documentclass{article}
\usepackage{graphicx}
\usepackage[margin=1.5cm]{geometry}
\usepackage{amsmath}
\usepackage{hyperref}

\begin{document}

\title{Week 6 Writing Activity: Chapter 5 of the \textit{Scientific Attitude}}
\author{Prof. Jordan C. Hanson (INTD100)}

\maketitle

\section{Practical Ways in Which Scientists Embrace the Scientific Attitude}

\begin{enumerate}
\item The author recounts an academic review study of published psychological results.  Two findings were important: (a) The rate of statistical errors made by authors increased if authors refused to shared data during publication, and (b) 96 percent of the errors were in the scientists' \textit{favor.}  Why do you suppose these findings occurred?  How is being willing to share data a part of the scientific attitude? \\ \vspace{3cm}
\item Define the term \textit{confirmation bias} in your own words. \\ \vspace{1cm} \\
\item Recall the story the author tells about cognitive research in logic puzzles involving cards and letters.  Researchers found that only 10 percent of individuals got the right answer, but that 80 percent of groups of individuals got the right answer.  (a) How do we show (statistically) that the groups are not just getting better results because people are differing to the smartest person in the group? (b) What do these results suggest about the grander notion of a \textit{scientific community}, as it relates to eliminating unintentional scientific error?

\end{enumerate}

\end{document}
