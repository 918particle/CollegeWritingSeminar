\documentclass{article}
\usepackage{graphicx}
\usepackage[margin=1.5cm]{geometry}
\usepackage{amsmath}
\usepackage{hyperref}

\begin{document}

\title{Week 7 Writing Activity: Chapter 5 of the \textit{Scientific Attitude}}
\author{Prof. Jordan C. Hanson (INTD100)}

\maketitle

\section{Technical Writing - Quantitative words and phrases}
\begin{enumerate}
\item Consider the following description of an experiment to measure the thickness of an ice shelf in Greenland:
\begin{quotation}
To measure the thickness of the Ross Ice Shelf, a radar-sounding system was mounted on an aircraft flown way above the ice surface.  Radar pulses were transmitted every few seconds as the plane flew quickly above the shelf, leading to one radar-sounding every few meters horizontally.  The radio pulses were less than one microsecond long, and had a frequency of 100 MHz.  The pulses travel downward at the speed of light until they pass through the snow surface.  Within the ice, the speed of radio pulses decreases.  The pulses reflect from the ocean beneath the ice, and return to the radar receiver on the plane.  From the total time of the sounding, the thickness of the ice was determined to be $550 \pm 6$ meters.
\end{quotation}
Write a paragraph below that describes the same measurement.  Replace ambiguous phrases with \textit{quantitative} words or phrases when possible.  It's ok to choose specific numbers.  The goal is to identify the phrases that need replacing. \\ \vspace{4cm}
\item Rewrite the following paragraph more quantitatively:
\begin{quotation}
In order to understand the whale shark behavioral patterns, it was necessary to measure the sea depth in the area of the Sea of Cortez where they gather annually.  A sound emitter was placed a little ways under the water surface, and emitted a 60 Hz sound wave with a pulse duration of 0.3 seconds, every several seconds.  A sound recorder was placed a little ways under the water surface, and recorded both the initial pulse and the echo from the bottom.  The time delay required for the sound to propagate down from the surface, reflect from the bottom, and return to the recorder was about 3.6 seconds.  The speed of sound in sea water is 1400 meters per second.  The implied depth is about 2500 meters.  The whale sharks are observed to be surfacing from very deep depths, but it is less likely that they spend a lot of time at deep depths.
\end{quotation}
\end{enumerate}
\end{document}
