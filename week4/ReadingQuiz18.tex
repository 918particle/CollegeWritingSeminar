\documentclass{article}
\usepackage{graphicx}
\usepackage[margin=1.5cm]{geometry}
\usepackage{amsmath}
\usepackage{hyperref}

\begin{document}

\title{Week 8 Writing Activity: Chapter 6 of the \textit{Scientific Attitude}}
\author{Prof. Jordan C. Hanson (INTD100)}

\maketitle

\section{The Scientific Attitude and Modern Medicine}

\begin{enumerate}
\item \textbf{The Barbarous Past}
\begin{enumerate}
\item Which of the following people is responsible for the theory of the four humours?
\begin{itemize}
\item A: Pasteur
\item B: Rush
\item C: Hippocrates
\item D: Galen
\end{itemize}
\item What are the four humours?  Provide several pieces of ``evidence'' for the theory of the four humours. \\ \vspace{1cm}
\item Describe the process of bloodletting, and comment on the effectiveness versus the risk. \\ \vspace{1cm}
\end{enumerate}
\item \textbf{The Dawn of Scientific Medicine}
\begin{enumerate}
\item Approximately when and where did the Scientific Revolution take place, for physics, astronomy, and chemistry? \\ \vspace{1cm}
\item And when did modern science begin to have an influence on medical practice in the United States? \\ \vspace{1cm}
\item In the late 1800s (19th Century), Louis Pasteur completed a set of experiments proving that food ``goes bad'' due to the growth of microscopic organisms called bacteria.  Keep in mind, \textit{he could not see them.}  Skeptics continued to practice medicine the old way, always asking ``Where are the little beasts?''  However, in 1879-1900 the bacteria responsible for major diseases were being ``discovered at the phenomenal rate of one a year.''  What technological breakthrough enabled these discoveries, and who was primarily responsible? \\ \vspace{1.5cm}
\end{enumerate}
\item \textbf{The Long Transition to Clinical Practice}
\item According to Thomas Kuhn, paradigm shifts often occur when ``the holdouts die and take their ideas with them.''  Give an example of this, or, list a few paradigm shifts where this social factor played a role.
\end{enumerate}

\end{document}
