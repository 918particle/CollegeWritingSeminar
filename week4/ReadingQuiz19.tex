\documentclass{article}
\usepackage{graphicx}
\usepackage[margin=1.5cm]{geometry}
\usepackage{amsmath}
\usepackage{hyperref}

\begin{document}

\title{Week 8 Writing Activity: Chapter 6 of the \textit{Scientific Attitude}}
\author{Prof. Jordan C. Hanson (INTD100)}

\maketitle

\section{Modern Medicine: The Long Transition to Clinical Practice}

\begin{enumerate}
\item According to Thomas Kuhn, paradigm shifts often occur when ``the holdouts die and take their ideas with them.''  Give an example of this, or, list a few paradigm shifts where this social factor played a role. \\ \vspace{2cm}
\item Recall the story of Abraham Flexner in 1908, who was commissioned by the American Medical Association and trhe Carnegie Foundation to visit all 148 medical schools in the United States.
\begin{itemize}
\item What are some of Flexner's findings regarding the medical school facilities? \\ \vspace{1cm}
\item What was the primary responsibility of the medical school professors?
\begin{itemize}
\item A: teaching
\item B: research
\item C: private practice
\item D: building laboratories
\end{itemize}
\item What were some of the recommendations of the Flexner Report to the AMA? \\ \vspace{2cm}
\item By 1922, the number of medical schools had fallen from 148 to 81.  Why did this happen? \\ \vspace{1cm}
\end{itemize}
\end{enumerate}

\section{The Fruits of Science}

\begin{enumerate}
\item Sketch how the discovery of penicillin took place in 1928.  \textit{Hint: the story of staphylococcus and mold ...}.  What part of the story was pure luck, versus the scientific attitude?
\end{enumerate}

\end{document}
